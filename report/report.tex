%resumé pour le project seanclog
%Auteur: Pieter Van Keymeulen
%Titre: Seanclog: Simple and Fancy timetracker

\documentclass[a4paper,11pt]{article}

%Een commando voor een mooie tabelhoofding
\newcommand{\head}[1]{\textnormal{\textbf{#1}}}

\bibliographystyle{plain}
\bibliography{tex}

\begin{document}
\title{Seanclog: Simple and Fancy timetracker}
\author{Pieter Van Keymeulen}
\maketitle

\section{Inleiding}

\subsection{Tijd. Planning. Al het andere}
Dit project heeft zijn oorsprong in mijn eigen persoonlijkheid. Als je het mijn
vrienden zou vragen of ik een ordelijke planner ben, zouden die allemaal
éénzelfde antwoord geven. Een gezamelijk neen. Omdat ik er vrijwel continu aan
herinnerd wordt, wou ik dit probleem aanpakken. Ik ben programmeur, en het
typische aan dit beroep is dat men ordelijk moet kunnen plannen. Een knelpuntje
voor mij dus. Ik kreeg vaak veel sofware toegewezen om mij hierbij te helpen.
Veel van deze software ben ik uiteindelijk beginnen gebruiken. Soms hou ik van
deze software, en soms haat ik deze software. Bij de software die ik haat heb
ik, als programmeur, altijd de neiging om deze te willen verbeteren.

Hierbij begon ik een specifieke haat te ontwikkelen voor timetrackers

\subsection{Het probleem?}
Een programmeur wilt natuurlijk op tijd betaald worden. De programmeur wordt
gehuurd, krijgt een opdracht, wordt betaald voor de gepresteerde uren. Maar hoe
moet je nu zo een factuur opstellen. Hoe weet een programeur nu hoe lang hij
voor een bepaald project heeft gewerkt en nog belangrijker, hoe weet de klant
dat hij een goede, eerlijke prijs betaald voor de afgeleverde software? Je kan
uren spenderen aan het invullen van spreadsheets. Die naar uw klanten sturen.
Verwachten dat uw klant deze zal opmerken. Dat terwijl de klant in kwestie zelf
onder honderden mails bedolven zit en uw mail niet eens opmerkt.

\subsection{Het licht gezien!}
%ici je raconte sur notre solution: Seanclog

\section{Interne structuur}

\subsection{Gebruikte software}

\begin{tabular}{c|c|c}
ik & ben & cool \\
dag & mooie & wereld
\end{tabular}

\subsection{RESTAPI} %hier heb ik wikipedia als bron gebruikt
Hier bespreken wij de RESTAPI \cite{restwiki}
Wij gaan voor deze applicatie gebruik maken van een RESTFUL api. Dit zorgt
ervoor dat we makkelijk meerdere clients kunnen maken die ook op op desktops
werken, telefoons, misschien zelfs een tv. De mogelijkheden zijn nagenoeg
eindeloos.

Een RESTApi bestaat uit de volgende componenten.
\begin{itemize}
  \item [Get] Een get request
  \item [Delete] Verwijder data
  \item [Put] Verander data
  \item [Post] Bewaar data
\end{itemize}

\subsection{Intern design}

\section{vormgeving}

%dit moet ik nog in subsecties opdelen

\end{document}