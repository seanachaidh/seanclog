%resumé pour le project seanclog
%Auteur: Pieter Van Keymeulen
%Titre: Seanclog: Simple and Fancy timetracker

\documentclass[a4paper,11pt]{article}

%Een commando voor een mooie tabelhoofding
\newcommand{\head}[1]{\textnormal{\textbf{#1}}}

\begin{document}
\title{Seanclog: Simple and Fancy timetracker}
\author{Pieter Van Keymeulen}
\maketitle

\section{Inleiding}

\subsection{Tijd. Planning. Al het andere}
Dit project heeft zijn oorsprong in mijn eigen persoonlijkheid. Als je het mijn
vrienden zou vragen of ik een ordelijke planner ben, zouden die allemaal
éénzelfde antwoord geven. Een gezamenlijk neen. Omdat ik er vrijwel continu aan
herinnerd wordt, wou ik dit probleem aanpakken. Ik ben programmeur, en het
typische aan dit beroep is dat men ordelijk moet kunnen plannen. Een knelpuntje
voor mij dus. Ik kreeg vaak veel software toegewezen om mij hierbij te helpen.
Veel van deze software ben ik uiteindelijk beginnen gebruiken. Soms hou ik van
deze software, en soms haat ik deze software. Bij de software die ik haat heb
ik, als programmeur, altijd de neiging om deze te willen verbeteren.

Hierbij begon ik een specifieke haat te ontwikkelen voor timetrackers

\subsection{Het probleem?}
Een programmeur wilt natuurlijk op tijd betaald worden. De programmeur wordt
gehuurd, krijgt een opdracht, wordt betaald voor de gepresteerde uren. Maar hoe
moet je nu zo een factuur opstellen. Hoe weet een programeur nu hoe lang hij
voor een bepaald project heeft gewerkt en nog belangrijker, hoe weet de klant
dat hij een goede, eerlijke prijs betaald voor de afgeleverde software? Je kan
uren spenderen aan het invullen van spreadsheets. Die naar uw klanten sturen.
Verwachten dat uw klant deze zal opmerken. Dat terwijl de klant in kwestie zelf
onder honderden mails bedolven zit en uw mail niet eens opmerkt.

\subsection{Het licht gezien!}
Na heel lang zoeken heb ik uiteindelijk de oplossing gevonden. Wanneer er geen
enkele goede oplossing ter beschikking staat rest de meesten slechts één ding.
Maak het zelf. Met dit project heb ik het doel om een timetracker te maken die
alle eigenschappen die ik aan timetrackers haat niet heeft. Hoofdzakelijk houdt
dit het volgende in.
\begin{itemize}
  \item Minder manueel invoeren
  \item Offline toegang
  \item Toegang vanop meerdere apparaten
  \item Een interface met zo weinig mogelijk afleiding
  \item Gemakkelijke filtering van ingevoerde tracks
\end{itemize}

\section{Interne structuur}

\subsection{Gebruikte software}
Voor dit project werd er bijna uitsluitend gebruik gemaakt van experimentele
technologie. Tegenwoordig worden webapplicaties vaak ontwikkeld op de zogenaamde
LAMP stack. Dit is een collectie software die meestal gebundeld gedistribueerd
wordt omdat ze tezamen de gebruiker in staat stellen tot het hosten van een
webapplicatie. De LAMP stack bestaat uit vier onderdelen.

\begin{description}
\item[Een Linux server] Dit is het besturingssysteem waarop de applicatie zal
draaien
\item[Apache] Dit is een daemon die het http protocol implementeert. Het stelt
ons in staat tot het ontvangen en het verwerken van http requests.
\item[Een MYSQL databank] Dit is te gebruiken databank bedoeld voor het oplaan
van data. Dit kan gaan om klantengegevens en productgegevens. Andere databanken
zoals postgresql zijn ook mogelijk, maar bij een LAMP stack gaat het specifiek
om MYSQL.
\item[PHP] De programmeertaal die gebruikt wordt voor het maken van de
applicatie.
\end{description}

Deze LAMP stack wordt al jaren op deze manier gebruikt. Het is vrij oude
technologie die al jaren meegaat en bijgevolg dus enorm betrouwbaar is. Het
heeft echter een aantal zwakke punten.

\begin{itemize}
  \item Het onderhoud van een linux server is een tijdrovende bezigheid.
  \item Apache is een modulair programma. Men kan functionaliteit toevoegen en
  verwijderen. Veel van de functionaliteit staat standaard aan of kan zelfs
  helemaal niet verwijderd worden. Dat terwijl de meesten niet zoveel
  functionaliteit nodig hebben. Daarbij kan het ook ervoor zorgen dat Apache
  onveiliger wordt en gemakkelijker is om in in te breken.
  \item MYSQL heeft veel limitaties. We leven in een wereld waarin data alsmaar
  meer aandacht krijgt. Bedrijven verzamelen, analyseren en gebruiken. Dit
  allemaal om hun klantenervaring zo veel mogelijk te kunnen verbeteren.
\end{itemize}

\subsubsection{Nodejs}
Nodejs is een javascript interpreter die in tegenstelling tot andere javascript interpreters niet gekoppeld is aan een webbrowser. Men kan hiermee javascripts uitvoeren alsof het gewone programma's zouden zijn. Dit heeft als neveneffect dat het ook gebruikt kan worden voor bijvoorbeeld server side scripting.

Dit is echter slecht één van de redenen waarom het een handige tool is voor het gebruik bij server side scripting. Tegenwoordig worden er alsmaar meer webapplicaties gebouwd. De bedrijven die deze webapplicaties bouwen hebben alsmaar meer klanten. Al die klanten moeten tegelijk afgehandeld worden. Computers hebben echter hun limieten. Wanneer een gebruiker een webpagina zou opvragen, betekend dat dat deze gebruiker een html bestand doorgestuurd zal moeten krijgen. Dat HTML bestand dient ingelezen en doorgestuurd te worden. Wanneer het besturingssysteem een bestand inleest betekent dat dat deze een zogezegde lock op dat bestand legt. Enkel het proces dat de leesbewerking heeft aangevraagd, mag het bestand inlezen. Van zodra dit proces gedaan heeft met lezen, wordt deze lock vrijgegeven en is het de beurt aan het volgende proces.

\subsubsection{MongoDB}
MongoDB is, in tegenstelling tot traditionele SQL databanken, een 
no-sql databank. Dit wil zeggen dat de data niet voorgesteld wordt door 
tabellen. Hoe de data dan wel voorgesteld wordt, wordt in de definitie 
van no-sql niet vervat. Over het algemeen kan men spreken over de 
volgende types nosql databanken

\begin{description}

\item[Column]
\item[Document]
\item[Key-value]
\item[Graph]

\end{description}


\subsubsection{Express}

\subsubsection{Angular}

\subsection{RESTAPI} %hier heb ik wikipedia als bron gebruikt
Hier bespreken wij de RESTAPI
Wij gaan voor deze applicatie gebruik maken van een RESTFUL api. Dit zorgt
ervoor dat we makkelijk meerdere clients kunnen maken die ook op op desktops
werken, telefoons, misschien zelfs een tv. De mogelijkheden zijn nagenoeg
eindeloos.

RestApi's worden doorgaans gebruikt voor het ophalen en het wegschrijven van
data. Deze data kunnen uiteenlopende dingen zijn. Afbeeldingen, blogartikels,
zelfs volledige objecten. Deze RestApi is meestal toegankelijk via het HTTP
protocol. Dit zorgt ervoor dat het makkelijker is om clients te maken die deze
API kunnen gebruiken omdat HTTP een alomvertegenwoordigd protocol heeft. Het is
een protocol dat standaard in praktisch iedere networking api geïmplementeerd is
en dat als dus danig door vele computers begrepen wordt.

Elk type data (of resource) wordt voorgesteld door een uniform resource
identifier (ofwel URI)

Een RESTApi bestaat uit de volgende componenten.
\begin{description}
  \item [Get] Een get request
  \item [Delete] Verwijder data
  \item [Put] Verander data
  \item [Post] Bewaar data
\end{description}

\subsection{Intern design}

\subsection{PhantomJS}

\section{vormgeving}

%dit moet ik nog in subsecties opdelen

\end{document}
